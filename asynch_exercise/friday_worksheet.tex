\documentclass[a4paper,11pt]{article}
\usepackage[margin=1in]{geometry}
\usepackage{enumerate}
\usepackage{xcolor}
 \usepackage{qtree}
\usepackage{fge}
\usepackage{tgtermes}

\usepackage{tikz}
\usepackage{tikz-qtree}

\usepackage{amsmath,amsthm, amssymb, latexsym}
\usepackage{graphicx, fancyhdr}
 \usepackage{stmaryrd}
 \usepackage{mathrsfs}
\usepackage{soul}
 \usepackage[utf8]{inputenc}
\usepackage[T1]{fontenc}

\usepackage{ot-tableau}
\usepackage{multicol}
\usepackage{enumitem}

\usepackage{tipa}
\usepackage{tipx}

\usepackage[sc]{mathpazo}
\usepackage[scaled=0.95]{helvet}
\usepackage{courier}
\usepackage{tgpagella}
\usepackage[euler]{textgreek}
 
\usepackage{gb4e}
\pagestyle{fancy}
\fancyhf{}
\fancyhead[C]{Ling201:02 \hfill 2025}
\fancyfoot[C]{\thepage}

\renewcommand{\theenumi}{\Alph{enumi}}


\begin{document}

\begin{center}
\Large{\textbf{Week 4 Recitation}}
\end{center}


\section{Morphological ambiguity}

\noindent{Are the following words ambiguous? (This is the practice we didn't get to in recitation last week.)}

\begin{itemize}
\item[$\rightarrow$] If yes, provide labeled \underline{bracketings} and \underline{trees} for both readings and explain what each reading means.
\item[$\rightarrow$] If no, explain why not.
\vspace{0.5cm}
\end{itemize}
\begin{exe}
\ex immeasurable 
\ex unwrappable
\ex unbearable
\end{exe}

\vspace{3cm}



\section{Syntax: constituents}

\noindent{\textcolor{blue}{\textbf{Please first review the slides I have posted in place of the recitation meeting!} Then use the four constituency tests from the slides to answer the following questions about each sentence.}}

\begin{exe}
\ex The happy student lives in New Brunswick \begin{xlist}
	\ex is "in New Brunswick" a constituent?
	\ex is "the happy" a constituent? 
	\ex is "student lives" a constituent?
\end{xlist}
\vspace{4cm}
\ex The tragedy upset the entire family. \begin{xlist}
	\ex is "the tragedy" a constituent?
\end{xlist}
\pagebreak
\ex They hid in the cave. \begin{xlist}
	\ex is "in the cave" a constituent? 
\end{xlist}
\vspace{3cm}
\ex The computer was very expensive. \begin{xlist}
	\ex is "computer was very" a constituent?
\end{xlist}
\vspace{3cm}
\ex The geese swam across the lake. \begin{xlist}
	\ex is "swam across" a constituent? 
\end{xlist}
\vspace{3cm}
\ex Mary gave a book to her sister. \begin{xlist}
	\ex is "a book to her sister" a constituent?
\end{xlist}
\vspace{3cm}
\ex Linda read a book about geology. \begin{xlist}
	\ex is "a book about geology" a constituent?
\end{xlist}
\end{exe}




\end{document}